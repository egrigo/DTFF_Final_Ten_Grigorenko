\documentclass {beamer}

\usetheme{Madrid}
\usepackage{url}

\title{Digital Tools for Finance}
\author [Ten, Grigorenko] {Elena Ten \and Elena Grigorenko}
\institute [UZH] {University of Zurich}
\date {15.12.2020}


\AtBeginSection
{
	\begin{frame}
	\frametitle{Overview}
	\tableofcontents[currentsection]

	\end{frame}
}


\begin{document}


\frame{\titlepage}

\section{Overview}
\begin{frame}
\frametitle{Overview}
This set of slides was produced to give an overview of different digital tools, used in our final project.\\
In general, we got experience with Git/Github, Slack, R, Python, SQL, LaTeX (Overleaf and Sublime Text editor) to elaborate the project. 

\end{frame}


% slides

\section{Version Control}
\begin{frame}
\frametitle{Version control}
Version control was implemented via Git. The commands were sent via command line and GitHub desktop.

\end{frame}


\section{Collaboration Tools}
\begin{frame}
\frametitle{Collaboration tools}

\begin{itemize}
\item We collaborated on the project, using Git.
\item We created a Slack channel and connected it to our Git repository.
\item Using Miro, we created the timeline of the project
\end{itemize}

%Insert Miro 
\begin{figure}[!h]
\includegraphics[scale=0.25]{miro}
\label{fig:miro}
\end{figure}


\end{frame}


\section{Writing with LaTeX}
\begin{frame}
\frametitle{LateX}
We configurated Sublime Text editor to use LaTeX.\\
\begin{itemize}
\item Using Sublime Text and TeXShop, we produced this set of slides as a beamer presentation.
\item Using Sublime Text and TeXShop, we produced the report (report.tex), that contains a table of contents, figures, tables and bibliography.
\item We elaborated the bibliography, having created an auxiliary file \path{/text/biblio.bib}.
\end{itemize}

\end{frame}

\section{Data Management}
\begin{frame}
\frametitle{Data management[1/2]}
For our project we performed a set of calculations in R (\path{/R/oil_stocks.Rproj}).\\
Specifically we calculated the cost of capital of oil companies with the following steps:
\begin{itemize}
\item Downloaded .csv files with stock and index data
\item Processed and filtered the data in R, using SQL
\item Used regression analysis to estimate historical betas
\item Produced LaTeX output in R
\item Built plots in R
\item Assembled the findings in LaTeX
\end{itemize}
\end{frame}

\begin{frame}
\frametitle{Data management[2/2]}
We also used Python for the analysis of stock prices.\\
Specifically we provided the overview of the industry stock prices  following the  steps:
\begin{itemize}
\item Called the data from Yahoo Finance API, using \emph{yfinance} library. The reasons for this choice of API
	\begin{itemize}
	\item free
	\item  large range of data
	\item  can be easily integrated into Python
	\end{itemize}
\item Uploaded necessary libraries for calling the data, analysis, plotting and prediction
\item Analysed Stock Prices
\item Predicted Stock Prices for 2 years
\end{itemize}

\end{frame}

\section{Analysis}
\begin{frame}
\frametitle{Analysis}
The analysis of Stock Prices was carried out in Python, using Jupyter Notebook.\\
Some of the libraries, that we used were:
\begin{itemize}
\item \emph{yfinance} to load data
\item \emph{prophet}  to predict stock prices, without constructing a sophisticated model
\item \emph{matplotlib} to plot the results
\end{itemize}
\end{frame}

\section{Visualisation}
\begin{frame}
\frametitle{Visualisation [1/2]}
To visualize the data we followed the strategies for effective data visualisation
\begin{itemize}
\item Follow Hierarchy of Visual Attention
\item Split the information to primary and secondary
\end{itemize}
The findings were visualised using RStudio and Jupyter Notebook

%Insert example
\begin{figure}[!h]
\includegraphics[scale=0.25]{SNP_SP}
\label{fig:snp1}
\end{figure}
\end{frame}


\begin{frame}
\frametitle{Visualisation[2/2]}
Limitations of the project due to Visualisation
\begin{itemize}
\item The prices of some companies distorted the graphs dramatically. We had to omit them for a better representation.
\item The graph on Return deviation of the industry was not included, as it was overwhelmed with colours, that layered on each other.
\end{itemize}
\end{frame}

\section{Knowledge Transfer}
%First slide on Knowledge transfer
\begin{frame}
\frametitle{Knowledge transfer}
With the means of R shiny we produced an interactive page, that processes user's input into graphs (\path{/R/data/app.R}).\\
The application allows user to choose one of the companies and time interval to visualize stock dynamics on a graph. The user may also choose an option to see the dynamics of S\&P 500 on the same graph.\\

% insert screenshot with input
\begin{figure}[!h]
\includegraphics[scale=0.25]{screenshot1}
\label{fig:ss1}
\end{figure}
\end{frame}

%Second slide on Knowledge transfer

\begin{frame}
\frametitle{Knowledge transfer}
Example of output:

% insert screenshot with output2
\begin{figure}[!h]
\includegraphics[scale=0.18]{screenshot2}
\label{fig:ss2}
\end{figure}

\end{frame}

% Contacts
\begin{frame}
\frametitle{Contacts}

Authors:\\
\vspace{5mm}
\textbf{Elena Ten}, 19-765-395, \href{mailto:elena.ten@uzh.ch}{elena.ten@uzh.ch}\\
\textbf{Elena Grigorenko}, 19-738-343, \href{mailto:elena.grigorenko@uzh.ch}{elena.grigorenko@uzh.ch} \\
\vspace{5mm}

\textbf{Thank you!}


\end{frame}



\end{document}