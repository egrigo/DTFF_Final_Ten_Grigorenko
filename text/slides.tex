\documentclass {beamer}

\usetheme{Madrid}

\title{Digital Tools for Finance}
\author [Ten, Grigorenko] {Elena Ten \and Elena Grigorenko}
\institute [UZH] {University of Zurich}
\date {15.12.2020}


\AtBeginSection
{
	\begin{frame}
	\frametitle{Overview}
	\tableofcontents[currentsection]

	\end{frame}
}


\begin{document}


\frame{\titlepage}

\section{Overview}
\begin{frame}
\frametitle{Overview}
This set of slides was produced to give an overview of different digital tools, used in our final project.\\
In general, we used Git/Github, Slack, R, Python, SQL language, Overleaf, Sublime Text editor to elaborate the project. 

\end{frame}


% slides

\section{Version Control}
\begin{frame}
\frametitle{Version control}
Version control was implemented with Git. The commands were sent via command line and GitHub desktop.

\end{frame}


\section{Collaboration Tools}
\begin{frame}
\frametitle{Collaboration tools}
We collaborated on the project, using Git.\\
Also we created a Slack channel and connected it to our Git repository.

\end{frame}


\section{Writing with LaTeX}
\begin{frame}
\frametitle{LateX}
We configurated Sublime Text editor to use LaTeX.\\
Using Sublime Text and Overleaf, we produced this set of slides and the report, located in the 'text' folder.
Also we elaborated the bibliography.


\end{frame}

\section{Data Management}
\begin{frame}
\frametitle{Data management}
We used SQL to work with our dataset on stock prices.\\
SQL queries were than used in our analysis in R.

\end{frame}

\section{Visualization}
\begin{frame}
\frametitle{Visualization}

\end{frame}


\section{Knowledge Transfer}
\begin{frame}
\frametitle{Knowledge transfer}
With the means of R shiny we produced an interactive page, that processes user's input into graphs.

\end{frame}






\end{document}